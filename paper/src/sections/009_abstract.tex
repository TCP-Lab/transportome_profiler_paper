\setcounter{footnote}{0} % Reset footnote counter

\begin{abstract}
    \large
    The transportome, the \textit{-omic} layer encompassing all \glspl{ict}, is crucial for cell physiology.
    It is therefore reasonable to hypothesize a role of the transportome in disease, and in particular in cancer.
    Here, we present the \gls{mtpdb}, a database collecting information on \glspl{ict}, and a pipeline that takes expression data and the \gls{mtpdb} as input to produce a broad overview of transportome dysregulation in cancer.
    The \gls{mtpdb} may prove useful for the study of the transportome in general, and the pipeline may be used to study the transportome in other diseases.
    Both tools are open source and can be found on GitHub at \href{https://github.com/TCP-Lab/MTP-db}{TCP-Lab/mtp-db} and \href{https://github.com/TCP-Lab/transportome_profiler}{TCP-Lab/transportome\_profiler}, under permissive licenses.
    We detect that the transportome is dysregulated in cancer, and that dysregulation patterns are shared among different cancer types.
    It is still unclear how these patterns are linked to cancer pathophysiology.
\end{abstract}

\setcounter{footnote}{0} % Reset footnote counter again
\glsresetall % To make glossary forget about the abstract.
