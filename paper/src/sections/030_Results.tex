\section{Results}

\begin{figure}
    \centering
    \includegraphics[width = 0.9\textwidth]{resources/images/generated/combined_heatmap.png}
    \caption{\small Results of \gls{gsea} about transportome dysregulation presented as a heatmap with the \gls{nes} values for each \gls{tgs} of interest (as rows) and the tested cohorts (as columns). Columns are clustered hierarchically. Opaque boxes are statistically significant (as reported by adjusted \textit{p}-values from \gls{gsea}) to an alpha threshold of $0.20$, while semi-transparent ones are not. Coordinates with a dot are significant to the same alpha threshold when considering absolute Wald-statistic values.}
    % Why don't we show the dendrogram above the colums?
    \label{fig:full_enrichment_heatmap}
\end{figure}

The enrichment matrix generated by the pipeline can be seen in Figure \ref{fig:full_enrichment_heatmap}.
The matrix provides a summarized overview of the over- or under-expression of each \gls{tgs} across all cohorts.
As each leaf node can be considered meaningfully different from any other node, the matrix represents an eagle's-eye-view of the dysregulation in the different cancer types.

In particular, some tumor types closely cluster with various signatures:
\begin{itemize}
    \item Kidney, colorectal, stomach, head and neck, and brain cancer show generalized dysregulation in many categories, and more specifically downregulation.
    \item Brain cancer shows downregulation or dysregulation in most categories.
    \item Pancreatic, esophageal, lung cancer and lymphoma show little to no deregulation, with the exception of a strong downregulation of proton \glspl{ict}.
    \item Prostate, testicular, skin, thyroid, bladder, female reproductive organs, adrenal gland, and breast cancers share a general upregulation of proton \glspl{ict}.
    \item Kidney cancer shows downregulation in a few categories, but shows significant dysregulation when looking at absolute metric values across almost all gene sets.
    \item The only cancer that shows a difference in expression of aquaporins is ovarian cancer.
\end{itemize}

Looking at the heatmap row-wise, we are able to appreciate other features:
\begin{itemize}
    \item With the exception of lymphoma and ovarian cancer, most cancer types show either no change or downregulation of pores.
    \item Transporters are generally dysregulated.
    In most cancer types they are upregulated, and seem downregulated in only Kidney, Head and Neck and Brain cancer.
    \item Not many cancers (just three, colorectal, prostate, and breast) show dysregulation in \gls{abc} transporters, but more show absolute dysregulation in the same category.
\end{itemize}

In conclusion, transportome genes are not uniformly dysregulated in cancer, but rather in a cancer-specific way.
However, functional considerations are limited by the fact that the gene sets are not completely mutually exclusive, and that they may not necessarily be functionally homogeneous.

Any implications that these overview has in the broader context of tumor physiology are still to be determined, but it is clear that the transportome may be a key player in cancer biology.
